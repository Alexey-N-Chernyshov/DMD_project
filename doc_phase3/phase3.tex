%%%%%%%%%%%%%%%%%%%%%%%%%%%%%%%%%%%%%%%%%
% Programming/Coding Assignment
% LaTeX Template
%
% This template has been downloaded from:
% http://www.latextemplates.com
%
% Original author:
% Ted Pavlic (http://www.tedpavlic.com)
%
% Note:
% The \lipsum[#] commands throughout this template generate dummy text
% to fill the template out. These commands should all be removed when 
% writing assignment content.
%
% This template uses a Perl script as an example snippet of code, most other
% languages are also usable. Configure them in the "CODE INCLUSION 
% CONFIGURATION" section.
%
%%%%%%%%%%%%%%%%%%%%%%%%%%%%%%%%%%%%%%%%%

%----------------------------------------------------------------------------------------
%	PACKAGES AND OTHER DOCUMENT CONFIGURATIONS
%----------------------------------------------------------------------------------------

\documentclass{article}

\usepackage{minted}
\usepackage{fancyhdr} % Required for custom headers
\usepackage{lastpage} % Required to determine the last page for the footer
\usepackage{extramarks} % Required for headers and footers
\usepackage[usenames,dvipsnames]{color} % Required for custom colors
\usepackage{graphicx} % Required to insert images
\usepackage{listings} % Required for insertion of code
\usepackage{courier} % Required for the courier font
\usepackage{lipsum} % Used for inserting dummy 'Lorem ipsum' text into the template
\usepackage{hyperref}

% Margins
\topmargin=-0.45in
\evensidemargin=0in
\oddsidemargin=0in
\textwidth=6.5in
\textheight=9.0in
\headsep=0.25in

\linespread{1.1} % Line spacing

% Set up the header and footer
\pagestyle{fancy}
\lhead{} % Top left header
\chead{\hmwkClass: \hmwkTitle} % Top center head
\rhead{\firstxmark} % Top right header
\lfoot{\lastxmark} % Bottom left footer
\cfoot{} % Bottom center footer
\rfoot{Page\ \thepage\ of\ \protect\pageref{LastPage}} % Bottom right footer
\renewcommand\headrulewidth{0.4pt} % Size of the header rule
\renewcommand\footrulewidth{0.4pt} % Size of the footer rule

\setlength\parindent{0pt} % Removes all indentation from paragraphs

%----------------------------------------------------------------------------------------
%	NAME AND CLASS SECTION
%----------------------------------------------------------------------------------------

\newcommand{\hmwkTitle}{Project\ Phase\ 3} % Assignment title
\newcommand{\hmwkDueDate}{Sunday,\ November\ 22,\ 2015} % Due date
\newcommand{\hmwkClass}{Data Modeling and Databases} % Course/class
%\newcommand{\hmwkClassTime}{10:30am} % Class/lecture time
\newcommand{\hmwkClassInstructor}{Qiang Qu} % Teacher/lecturer
\newcommand{\hmwkAuthorName}{Alexey Chernyshov, Vladislav Kravchenko, Murad Magomedov} % Your names

%----------------------------------------------------------------------------------------
%	TITLE PAGE
%----------------------------------------------------------------------------------------

\title{
\vspace{2in}
\textmd{\textbf{\hmwkClass:\ \hmwkTitle}}\\
\normalsize\vspace{0.1in}\small{Due\ on\ \hmwkDueDate}\\
\vspace{0.1in}\large{\textit{\hmwkClassInstructor\ }}
\vspace{3in}
}

\author{\textbf{\hmwkAuthorName}}
\date{} % Insert date here if you want it to appear below your name

%----------------------------------------------------------------------------------------

\begin{document}

\maketitle

%----------------------------------------------------------------------------------------
%	TABLE OF CONTENTS
%----------------------------------------------------------------------------------------

%\setcounter{tocdepth}{1} % Uncomment this line if you don't want subsections listed in the ToC

\newpage
\tableofcontents
\newpage

%----------------------------------------------------------------------------------------
\section{Phase 3. DBMS}

\subsection{Introducing}
According to Project Phase 3 requirements we had to replace the relational database with our own implementation of a database. So we have done it, including implementation of query operators using python programming language.

\subsection{Functionality}
At phase 3 all SQL code was replaced with calls to the query operators. So now our database provides methods to query data, to insert new records and to update existing records. \\

\textbf{For example, SQL code of Select:}

\begin{minted}{sql}
cur.execute("""SELECT id, name, institute FROM author WHERE id=%s;""", (id, ))
\end{minted}

\textbf{Replaced with:}

\begin{minted}{python}
qr_author = qp.getFromTable('author', )
qr_res = qp.getFromTable('author', ('id', id))
qr_res = qr_res.project('id', 'name', 'institute')
qr_res = qr_res.sort('id')
\end{minted}
~
\\
\textbf{Delete:}

\begin{minted}{sql}
cur.execute("""DELETE FROM article_author WHERE author_id=%s""", (id, ))
\end{minted}

\textbf{Replaced with:}

\begin{minted}{python}
qp.deleteFromTable('article_author', ('author_id', author_id))
\end{minted}
~
\\
\textbf{Update:}

\begin{minted}{sql}
cur.execute("""UPDATE author SET id=%s, name=%s, institute=%s
            WHERE id=%s""", (id, name, institute, id))
\end{minted}

\textbf{Replaced with:}

\begin{minted}{python}
qp.deleteFromTable('author', ('id', id))
qp.addToTable('author', ('id', id), ('name', name), ('institute', institute))
\end{minted}
~
\\
\textbf{Insert:}
\begin{minted}{sql}
cur.execute("""INSERT INTO reference (from_id, to_id)
            VALUES (%s, %s)""", (id, article_id))
\end{minted}

\textbf{Replaced with:}

\begin{minted}{python}
qp.addToTable('reference', ('from_id', id), ('to_id', article_id))
\end{minted}
~
\\
\\
\textbf{Group by:}

\begin{minted}{python}
qres.groupBy('id', 'name')
\end{minted}

\begin{verbatim}
Our database also answers queries that join two tables, for example join of tables 
article_author and article:
\end{verbatim}

\textbf{Join:}

\begin{minted}{python}
qr_article_author = qp.getFromTable('article_author', ('author_id', author_id))
qr_article = qp.getFromTable('article')
qr_res = qr_article_author.join(qr_article, 'article_id', 'id')
qr_res = qr_res.project('article_id', 'paper_title')
qr_res = qr_res.sort('article_id')
\end{minted}

One of the requirements of Phase 3 was to use the iterator model to implement query operators, so that operators should pull tuples from underlying operators using next() calls. \\
So, in our case, class qres (query result) is iterable. It is possible to use qres.next or next(qres) to iterate through results.
In our DBMS it is possible to index on Primary keys with the ability to add index on other attributes. \\
And according to Project limitations we are using a single file for entire DBMS. The Database file is created using psqlloader.py that imports data from Postgres.

\newpage
\subsection{Disk Page Organisation}
~
\newline
\begin{figure}[h!]
  \centering
      \includegraphics[width=17cm]{dpo.png}
  \caption{Disk Page Organisation}
\end{figure}

\newpage
\subsection{Class diagram}
~
~
\begin{figure}[h!]
  \centering
      \includegraphics[width=17cm]{cd.png}
  \caption{Class Diagram}
\end{figure}
~
~

~
\\

\newpage
\begin{thebibliography}{1}

\bibitem{def_tornadobook} 
\ Concepts of Database Management. Philip J. Pratt, Joseph J. Adamski. Cengage Learning, 2008. \\ ISBN: 1-4239-0147-9

\bibitem{def_tornadobook} 
\ Database management systems. Raghu Ramakrishnan, Johannes Gehrke. McGraw-Hill., 2003. \\ ISBN: 0-07-246563-8-ISBN 0-07-115110-9 (ISE)

\bibitem{def_tornadobook} 
\ A First Course in Database Systems. Jeffrey D. Ullman, Jennifer Widom. Pearson Education., 2008. \\ ISBN: 978-0-13-502176-7


\end{thebibliography}

\end{document}